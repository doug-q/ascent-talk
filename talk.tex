% Created 2024-06-27 Thu 15:24
% Intended LaTeX compiler: pdflatex
\documentclass[presentation]{beamer}
\usepackage[utf8]{inputenc}
\usepackage[T1]{fontenc}
\usepackage{graphicx}
\usepackage{longtable}
\usepackage{wrapfig}
\usepackage{rotating}
\usepackage[normalem]{ulem}
\usepackage{amsmath}
\usepackage{amssymb}
\usepackage{capt-of}
\usepackage{hyperref}
\usepackage{minted}
\usetheme{Madrid}
\author{\href{mailto:douglas.wilson@quantinuum.com}{Douglas Wilson}}
\date{\today}
\title{Dataflow Analysis of \texttt{Hugr}​s with \texttt{ascent}}
\hypersetup{
 pdfauthor={\href{mailto:douglas.wilson@quantinuum.com}{Douglas Wilson}},
 pdftitle={Dataflow Analysis of \texttt{Hugr}​s with \texttt{ascent}},
 pdfkeywords={},
 pdfsubject={},
 pdfcreator={Emacs 29.3 (Org mode 9.8)}, 
 pdflang={English}}
\begin{document}

\maketitle
\begin{frame}{Outline}
\tableofcontents
\end{frame}

\section{Datalog}
\label{sec:org7223913}
\begin{frame}<1->[label={sec:orgee00de9},fragile]{Example}
 \begin{minted}[]{rust}
ascent! {
   relation edge(i32, i32);
   relation path(i32, i32);

   path(x, y) <-- edge(x, y);
   path(x, z) <-- edge(x, y), path(y, z);
}
\end{minted}
\begin{onlyenv}<1>
\begin{itemize}
\item Here is an example that, given the edges of a directed graph, computes whether a path exists between any two nodes.

\item A datalog program is a collection of \emph{Horn Clauses}(Horn, Alfred, 1951).

\item The most mature open source implementation of Datalog is \emph{souffle}(Scholz, Bernhard and Jordan, Herbert and Suboti\'{c}, Pavle and Westmann, Till, 2016,  Herbert Jordan and Bernhard Scholz and Pavle Suboti{\'c}, 2016). It works by generating \texttt{C++} code from a Datalog program.
\end{itemize}
\end{onlyenv}
\begin{onlyenv}<2>
\begin{itemize}
\item Think of relations as sets of facts.
\item A datalog solver computes all true facts from a set of initial facts.
\item A datalog solver mutates relations as it iterates
\item This mutation is \emph{monotone}: once a fact exists, it will always exist.
\end{itemize}
\end{onlyenv}
\begin{onlyenv}<3>
\texttt{ascent} (Sahebolamri, Arash and Gilray, Thomas and Micinski, Kristopher, 2022) is an implementation of Datalog via \texttt{rust} proc-macro.

\begin{minted}[]{rust}
fn main() {
   let mut prog = AscentProgram::default();
   prog.edge = vec![(1, 2), (2, 3)];
   prog.run();
   println!("path: {:?}", prog.path);
}
\end{minted}
\end{onlyenv}
\begin{onlyenv}<4>
Datalog is tractable to solve because of \emph{semi-naive evaluation}. Once a fact exists it always exists, therefore during an iteration we know that any new fact we find must depend on a fact that we discovered in the previous iteration.
\end{onlyenv}
\end{frame}
\begin{frame}[label={sec:org276e755},fragile]{Conditions and Generative Clauses}
 \begin{minted}[]{rust}
ascent! {
   relation node(i32, Rc<Vec<i32>>);
   relation edge(i32, i32);

   edge(x, y) <--
      node(x, neighbors),
      for &y in neighbors.iter(),
      if x != y;
}
\end{minted}
\end{frame}
\begin{frame}[label={sec:orge56f7f8},fragile]{Negation and Aggregation}
 \begin{minted}[]{rust}
use ascent::aggregators::*;
type Student = u32;
type Course = u32;
type Grade = u16;
ascent! {
   relation student(Student);
   relation course_grade(Student, Course, Grade);
   relation avg_grade(Student, Grade);

   avg_grade(s, avg as Grade) <--
      student(s),
      agg avg = mean(g) in course_grade(s, _, g);
}
\end{minted}
\end{frame}
\begin{frame}[label={sec:orgdd26c00}]{Lattices}
A \emph{Lattice} is a partial order equipped with:
\begin{itemize}
\item a binary operation \emph{join}, or least upper bound;
\item a binary operation \emph{meet}, or greatest lower bound.
\end{itemize}

A \emph{Bounded Lattice} has a unique maximum element, called \emph{top} or \(\top\) and a unique minimum element called \emph{bottom} or \(\bot\).
\end{frame}
\begin{frame}[label={sec:orgbadc3c1},fragile]{Lattices In \texttt{ascent}}
 \begin{minted}[]{rust}
ascent! {
   lattice shortest_path(i32, i32, Dual<u32>);
   relation edge(i32, i32, u32);

   shortest_path(x, y, Dual(*w)) <-- edge(x, y, w);

   shortest_path(x, z, Dual(w + l)) <--
      edge(x, y, w),
      shortest_path(y, z, ?Dual(l));
}
\end{minted}

\begin{itemize}
\item a member of a \emph{lattice} \((k, L)\) is a fact that implies that \((k,l), l \leq L\) is a fact.
\item \texttt{Dual} is a newtype wrapper that swaps \emph{meet} and \emph{join}. Unfortunately \texttt{longest\_path} will fail to terminate on any graph with cycles.
\end{itemize}
\end{frame}
\begin{frame}[label={sec:orge0142ed},fragile]{Pluggable data structures in \texttt{ascent}}
 An eqivilence relation:
\begin{minted}[]{rust}
ascent! {
    relation rel(u32, u32);
    rel(a,b) <- rel(b, a)
    rel(a,c) <- rel(a, b), rel(b, c)
}
\end{minted}
will create \(N^2\) facts.

We can store those facts using only \(N\) using a \emph{union-find} data structure:
\begin{minted}[]{rust}
ascent! {
    #[ds(rels_ascent::eqrel)]
    relation rel(u32, u32);
    // ...
}
\end{minted}
\end{frame}
\begin{frame}[label={sec:org339d530},fragile]{Pluggable data structures in \texttt{ascent}}
 (Sahebolamri, Arash and Barrett, Langston and Moore, Scott and Micinski, Kristopher, 2023) describes an interface to store relations in user-defined data structures.

Users implement several macros, which are then expanded by the \texttt{ascent!} macro.
\end{frame}
\begin{frame}[label={sec:orgeb6798e}]{Performance}
TODO
\end{frame}
\begin{frame}[label={sec:org8fc791a}]{Why use a Datalog solver at all?}
\begin{itemize}
\item Split one hard problem into two slightly easier problems:
\begin{itemize}
\item A Datalog solver
\item A Datalog program
\end{itemize}
\end{itemize}

Separate the specification and the implementation of your problem.
\end{frame}
\begin{frame}[label={sec:orgb3662ae},fragile]{Downsides of \texttt{ascent}}
 \begin{itemize}
\item the proc-macro implementation seems to make it difficult to write an extensible tool.
\end{itemize}
\end{frame}
\section{An important quantum optimisation problem}
\label{sec:org9f557be}
\begin{frame}[label={sec:org337a0f6},fragile]{Can't optimise this}
 \begin{minted}[]{python}
@guppy
def circuit(q: Qubit) -> Qubit:
    i = 0
    while i < 2:
        u = h(Qubit())
        if i % 2 == 0:
            q, u = cx(q, u)
        else:
            q, u = cy(q, u)
        i = i + 1
        u.free()
    return q
\end{minted}
\end{frame}
\begin{frame}[label={sec:org4a345aa},fragile]{Can optimise this}
 \begin{minted}[]{python}
@guppy
def circuit(q: Qubit) -> Qubit:
  u1, u2 = (Qubit(), Qubit())
  u1 = h(u1)
  q, u1 = cx(q, u1)
  u2 = h(u2)
  q, u2 = cy(q, u2)
  u1.free()
  u2.free()
  return q
\end{minted}
\end{frame}
\begin{frame}[label={sec:org09a7548}]{Dataflow analysis}
\emph{Dataflow Analysis} is a general technique for static program analysis. \emph{SSA} is particularly well suited for this:
\begin{itemize}
\item Choose a \emph{Lattice} type with a \emph{bottom}.
\item Assign \(\bot\) to each edge. (i.e. each ``value'' in an SSA graph)
\item Define a \emph{transfer function} that takes a node and Lattice values for each of its edges, and returns Lattice values for each of its edges.
\item Apply the transfer function to each node and mutate the Lattice values for each of its edges by \emph{joining} with the result of the transfer function.
\item Iterate the previous step until you reach a fixed point.
\end{itemize}
\end{frame}
\begin{frame}[label={sec:org1ba4bb6},fragile]{Liveness Analysis}
 Lattice: Define \(\bot\) to be \emph{Dead} and \(\top\) to be \emph{Live}.

Transfer function: The arguments of \texttt{return} are \emph{Live}, the arguments of any node with \emph{Live} results are \emph{Live}.

\begin{minted}[]{python}
@guppy
def circuit(q: Qubit, theta: float) -> Qubit: # theta is dead
    theta = -theta
    return q # q is live
\end{minted}
\end{frame}
\begin{frame}[label={sec:orgaeaade6},fragile]{Constant Value Propagation}
 \begin{onlyenv}<1->
Define the following lattice:

\begin{minted}[]{rust}
enum ConstantValue { Bottom, Value(u64), Top }
fn join(lhs: ConstantValue, rhs: ConstantValue) -> ConstantValue {
  match (lhs, rhs) {
    (Bottom, x) => x,
    (x, Bottom) => x,
    (Value(x), Value(y)) if x == y => Value(x),
    _ => Top
  }
}
\end{minted}
\end{onlyenv}
\begin{onlyenv}<1>
Transfer Function: this is constant folding.

Consider:
\begin{itemize}
\item add(Value(x),Value(y))
\item mult(Top,Value(0))
\end{itemize}
\end{onlyenv}
\begin{onlyenv}<2>
\begin{itemize}
\item After iterating the transfer function, if any node has a \(\bot\) input, then that node is \emph{Unreachable}. Perhaps it is in the \emph{else} branch of an always-true \emph{if} statement.

\begin{itemize}
\item Theorem: Values of type \emph{The sum of zero variants} (also called \(\bot\)). Will always be assigned the lattice value \(\bot\).
\end{itemize}

\item Liveness analysis should only mark the inputs of \emph{Reachable} \texttt{return} statements.
\end{itemize}
\end{onlyenv}
\end{frame}
\begin{frame}[label={sec:org9d7d180},fragile]{PartialValue}
 \begin{minted}[]{rust}
enum PartialValue {
    Bottom,
    Value(hugr::ops::Value),
    PartialSum(HashSet<usize, Vec<PartialValue>>),
    Top
}
\end{minted}

\begin{itemize}
\item \emph{PartialValue} refines the idea of \emph{ConstantValue} to try a little bit harder to not \emph{join} to \(\top\).

\item \emph{PartialSum} keeps track of which variant it might be, and what those variant's values might be.

\item In a \emph{Hugr} all non-function call control flow is controlled by the \emph{tag} of a variant.
\begin{itemize}
\item Conditional
\item TailLoop
\item CFG (arbitrary control flow graph)
\end{itemize}

\item Let's look at \texttt{dataflow.rs} in \url{https://github.com/CQCL/hugr/tree/doug/const-fold2-talk}
\end{itemize}
\end{frame}
\begin{frame}[label={sec:org90ed210},fragile]{Loop unrolling}
 Imagine a function:

\begin{minted}[]{rust}
/// Apply constant value propagation to the inner DFG of a
/// dataflow parent, using `in_values` as the values for the
/// `Input` node. Returns the values for the `Output` node
pub fn cvp_dataflow_parent(hugr: &Hugr, dataflow_parent: Node,
        in_values: Vec[PartialValue]) -> Vec[PartialValue];
\end{minted}

We can use this to unroll a \texttt{TailLoop} node \texttt{tl}:
\begin{itemize}
\item Do constant value propagation on the \texttt{hugr: Hugr}, and retrieve the input values for the \texttt{tl}.
\begin{itemize}
\item Call \texttt{let out\_values = cvp\_dataflow\_parent(hugr, tl, in\_values);}.
\item if \texttt{!out\_values[0].supports\_tag(1)} then the tailloop is proven to iterate at least once.
\item Create a \texttt{DFG} before \texttt{tl}, containing a copy of \texttt{tl}, wired up to the old inputs of \texttt{tl}, and with its
outputs becoming the new inputs of \texttt{tl}.
\item set \texttt{in\_values = out\_values} and iterate.
\end{itemize}
\end{itemize}
\end{frame}
\section{Conclusion}
\label{sec:orgdb585cc}
\begin{itemize}
\item Dataflow Analysis is a useful tool and Hugrs are well suited for it to be
directly applied. (Heidemann, ????)
\item Constant Value Propagation is strong enough to unroll loops in Hugr.
\item It is not clear whether \texttt{ascent} is an appropriate tool. How can write modular interdependent analases?.
\end{itemize}
\end{document}
